
%   Packages, math formatting

\documentclass{article}

%T% use a package with the \usepackage{packageName} directive
\usepackage{amsmath}

\begin{document}

%T% create an equation with the \begin{equation}, \end{equation} tags
    \begin{equation}
        f_1(x) = y
    \end{equation}
    
%T% create unnumbered amsmath equation with the 
%T% \begin{equation*}, \end{equation*} tags
    \begin{equation*}
        f_2(x) = N*n
    \end{equation*}

%T% create an inline equation by enclosing it in $ dollar signs
    textPre $equation = a$ textPost

%T% create several equations with the align* or align tags
%T% the alignment is done around the & ampersand
    \begin{align*}
        &f_1\\
        f_2&\\
        f&_3
    \end{align*}

%T% typesetting mathematical notation is as follows,
%T% sum +, subtraction -, multiplication *, division /, power ^
%T% fractions \frac{num}{den}, integrals \int_infLim^supLim, 
%T% square root \sqrt{}, summation \sum_{}^{}, infinity symbol \infty
    \begin{align}
        a + b - c = x * y / z ^z\\
        \frac{a}{b} = \int_x^y Z\\
        \sqrt{x_1}\\
        \sum_{u \in U_{s}}^{\infty} Z_{i,u,s} = 1
    \end{align}
%T% less than and greater than or equal to with \leq, \geq
%T% insert spaces in equations with \quad and \qquad
    $a \leq \quad x \qquad \geq b$

%T% put a tilde on top of a text with \tilde{text}
    $\tilde{A}$

%T% put a tilde that covers the whole text with \widetilde{text}
    $\widetilde{long Text}$
        
%T% create special symbols like the forall symbol, in set, ellipsis,
%T% curly braces
    $\forall \in \dots \{ \}$

%T% define a function by parts with the cases environment
    \begin{align}
        \begin{cases}
            val1,  & range1\\
            v2,    & range2
        \end{cases}
    \end{align}

%T% grouping symbols

% |--------------------------------------------------\
%T% grouping symbols such as parentheses, brackets, braces, etc., are commonly paired as a left, right pair. This pairing is ensured with the left, right commands

% SYNTAX \left\grouping_symbol1 text1 \right\grouping_symbol2
%T% the grouping symbol is placed after the left, right commands, text1 is the text between the grouping symbols, \grouping_symbol1 need not be the same as \grouping_symbol2

    $\left\langle str1 \right\rangle$
% |--------------------------------------------------/

%T% create matrices with the matrix tag, within an align environment
%T% scale brackets and parentheses with \left(, \right), \left[, \right]
    \begin{align}
        \left[
        \begin{matrix}
            1 & x & y\\
            z & & t*\lambda
        \end{matrix}    
        \right]
    \end{align}

%T% use ceil and floor delimiters with \lceil, \rceil, \lfloor, \rfloor
    $\lceil k \rceil = \lfloor\frac{N}{2}\rfloor$

% |--------------------------------------------------\
%T% the array environment is used with extra arguments to set the alignment, the arguments can be
%T%     l for left
%T%     c for center
%T%     r for right
%T%     | for line separator
%T%     : for dotted line separator (in Katex)

% SYNTAX array creation
% \def\arraystretch{N1}
% \begin{array}{l|c|r}
%     str1 & str2 & str3\\
%     str4 & str5 & str6
% \end{array}
%T% N1 is a number with the amount of stretch for the array

    $\def\arraystretch{3}
    \begin{array}{l|c|r}
        a & b & c\\
        d & e & f
    \end{array}$    
% |--------------------------------------------------/

%T% create a single multiline equation inside an align environment,
%T% with the split environment
    \begin{align}
        \begin{split}
            \widetilde{Tf}_{i,s} + M*(2-Z_{i,u,s}-&Z_{j,u,s}) \leq 
            \widetilde{Ts}_{j,s} \\
            \forall i \in I, j \in I, i < j,& s \in S, u \in U_{s}    
        \end{split}
    \end{align}

%T% the alignedat environment allows the alignment of more than two columns, the amount of columns is specified via argument
    $\begin{alignedat}{3}
        ab& &dc \quad &e\\
        &ba \quad cd& &e
    \end{alignedat}$

\end{document}