
%   PSTricks

%T% draw geometry with PSTricks

%T% compile with xelatex, it gives errors with pdflatex

\documentclass{article}

%T% use the pst-eucl package to create geometric images
\usepackage{pst-eucl}

\begin{document}

%T% the geometry is created inside the pspicture tag
\begin{pspicture}[showgrid=true](-2,-2)(3,3)


%T% Points section

%T% points can be created with \pstGeonode(p1x,p1y){p1Name}(p2x,p2y){p2Name}

%T% PointSymbol=symbol, symbol can be: pentagon*, otimes, oplus, diamond, 
%T% square, o, *, +, x, triangle, |, etc.
%T% the asterisk in pentagon* means filled

%T% create lines between points with CurveType=type, type can be: polyline,
%T% polygon, curve

%T% change the options of each point independently with opt={p1,p2,p3}
%T% [PosAngle=angle,PointNameSep=separation,dotscale=scale,fillcolor=color,
%T% linecolor=color,PointName=name]
    \pstGeonode[PointSymbol={|,x,*},CurveType=curve,PosAngle={76,0,-30},
    PointName=default](4,-13){Pb}(2,-11){SecondNameLabel}(-1,-10){Tp}

%T% move a node creating a new one with \pstMoveNode(dx,dy){NodeIni}{NodeEnd}
    \pstMoveNode(1,-2){Tp}{MvTp}

%T% create a line with \pstLineAB[opts]{nodeA}{nodeB}
    \pstLineAB[linecolor=blue]{Tp}{MvTp}

%T% create a simpler line with \psline(xIni,yIni)(xEnd,yEnd)
    \psline(-1,3)(2,1)

    \pstGeonode(-2,1){O}(-3,2){I}(0.5,2){J}
%T% extend the line like an axis with [nodesep=num]
    \pstLineAB[linecolor=blue,nodesep=5]{O}{J}

%T% create multiple parallel lines, like tick marks, or a grid
%T% with \multips(dxIni,dyIni)(dxR,dyR){numOfRepeatitions}{lineToRepeat}
%T% where dxR, dyR are the distances between repetitions
    \multips(-1,-2)(0,-0.5){3}{\psline(-1,3)(2,1)}

%T% set options for subsequent geometry with \psset{opts}
    \psset{linestyle=dotted}%

%T% create a set of points in a transformed coord system 
%T% with \pstOIJGeonode[opts](p1x,p1y){p1}{O}{I}{J}(p2x,p2y){p2}
%T% where O is origin, I is xAxisUnit, J is yAxisUnit transformed
    \pstOIJGeonode[CurveType=polyline,linecolor=red]
    (1,1){A_1}{O}{I}{J}(-2,1){B_1}(2,-0.5){C_1}

\end{pspicture}

\newpage
\begin{pspicture}[showgrid=true]


%T% Segment section

    \pstGeonode[PosAngle={-35,70}](1,2){A_2}(6,4){B_2}
%T% create a marked segment with \pstSegmentMark[opts]{p1}{p2}
%T% the mark can be changed with [SegmentSymbol=markSym,
%T% MarkAngle=num,MarkHashLength=num,MarkHashSep=num]
%T% values of markSym are: pstslashhh (up to 3 h), MarkHashhh,
%T% MarkCross, MarkArrowww
    \pstSegmentMark[SegmentSymbol=MarkHashhh,
        MarkAngle=190,MarkHashLength=1,MarkHashSep=1]{A_2}{B_2}

%T% measure a segment with \pstLabelAB[opts]{p1}{p2}{measureLabel}
    \pstLabelAB[offset=15pt,linestyle=solid,arrows=|<->|]
        {A_2}{B_2}{$\sqrt{x^2+y^2}$}
    
\end{pspicture}

\newpage
\begin{pspicture}[showgrid=true]
    
    \psset{PointSymbol=none}
    \pstGeonode[PointName={default,default,none,default}]
        (0,0){A_3}(1,2){B_3}(7,1){C_3}(4,3){D_3}
    \pstLineAB{A_3}{B_3}\pstLineAB{B_3}{C_3}\pstLineAB{C_3}{D_3}


%T% Angle section

%T% create a right angle symbol 
%T% with \pstRightAngle[opts]{side1}{vertex}{side2}
    \pstRightAngle[RightAngleSize=1]{A_3}{B_3}{C_3}
    
%T% create a general angle 
%T% with \pstMarkAngle[opts]{side1}{vertex}{side2}{label}
    \pstMarkAngle[MarkAngleRadius=0.3,LabelSep=0.5,LabelAngleOffset=97,
        arrows=->,MarkAngleType=double,Mark=MarkHash]{B_3}{C_3}{D_3}{$\theta$}

\end{pspicture}

\end{document}