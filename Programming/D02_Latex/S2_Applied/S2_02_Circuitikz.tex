
%   Circuitikz

\documentclass{article}

\usepackage{tikz}
%T% use the package circuitikz to draw circuits
\usepackage{circuitikz}

\begin{document}

\begin{figure}

%T% draw the circuit with the circuitikz tag
    \begin{circuitikz}
        
%T% draw the circuit with lines, create components inside the options
%T% create a voltage source with [V,v=voltageQuantity]
%T% create a resistor with [R=resistanceQuantity]
%T% create a capacitor with [C=capacitanceQuantity]
%T% create an inductor with [L=inductanceQuantity]
%T% add a current or voltage arrow with [(i|v|l)(POS)=symbolText],
%T% in (i|v|l) select only one, i for current, v for voltage, l for label
%T% where (POS) indicates positioning though the symbols _ (down), ^ (up)
%T% < (left), > (right)
        \draw (0,0) to[V,v<=5V] (0,2) to[L=k] (3,2) to[R=1$\Omega$] (3,0)
        to[C=3mF,i<_=i] (0,0);

%T% establish that a line is a short with [short]
%T% change the endings of a line in the options, e.g. [o-*]
        \draw (4,4) to[short,o-*] (7,4);

%T% create a ground connection with node[ground]{GNDName}
        \draw (0,-1) to (0,-2) node[ground]{GND};

%T% create a BJT with
        \draw (2,-2) node[pnp,rotate=90](bjt1){BJT} (bjt1.base) node {B}
        (bjt1.collector) node {Coll} (bjt1.emitter) node {emi};

    \end{circuitikz}
\end{figure}

\end{document}