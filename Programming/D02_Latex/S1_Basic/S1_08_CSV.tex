
%   CSV

%T% CSV stands for Comma Separated Values

%T% CSV data import

\documentclass{article}

%T% use the pgfplotstable package to use CSV files for tables
\usepackage{pgfplotstable}

%T% use the tikz package to create tikz pictures
\usepackage{tikz}

\begin{document}

\begin{table}

%T% works for one page tables
%T% inside a table select the CSV file and the read parameter options
%T% with \pgfplotstabletypeset[options]{path/to/csvFile}
    \pgfplotstabletypeset[
%T% set the column separator
      col sep=comma,
%T% set the way columns will be displayed
      display columns/0/.style={
		column name=Col1 Header, % name of first column
		column type={r|},string type},  % use siunitx for formatting
      display columns/1/.style={
		column name=Col2 Header,
		column type={|l},string type},
    ]{Section1_8Aux.csv}

\end{table}

\begin{figure}
    
%T% create a figure for the plot with the tikzpicture tag
    \begin{tikzpicture}
        
%T% create the axes for the plot with the axis tag
        \begin{axis}[width=\linewidth]

%T% plot the actual data with \addplot table[opts]{path/to/csv};
%T% in opts x and y are the header names in the CSV file
            \addplot table[x=col1,y=col2,col sep=comma]{Section1_8Aux.csv};
        \end{axis}
    \end{tikzpicture}

\end{figure}

\end{document}