
%   Source code

%T% formatting as source code

\documentclass{article}

%T% use the listings package to display formatted source code
\usepackage{listings}

%T% use the color package to give colors to keywords and comments
\usepackage{color}

%T% set the programming language format with \lstset{params}
\lstset{language=C,keywordstyle=\color{blue},
    commentstyle=\color{green}}

\begin{document}

%T% write source code with the lstlisting tag, 
%T% use mathescape=true to escape math within dollar symbols $$
%T% use literate={á}{{\'a}}1 {é}{{\'e}}1 {í}{{\'i}}1 {ó}{{\'o}}1 {ú}{{\'u}}1 to map letters with diacritics in a way that can be used inside lstlisting, since this environment can't use regular utf-8 encoding
\begin{lstlisting}[mathescape=true, literate={á}{{\'a}}1 {é}{{\'e}}1]
    int a_1 = 10;
    $\widetilde{K} = \frac{n}{2}$
    printf("debug string");
\end{lstlisting}

%T% input a source code file with \lstinputlisting{path/to/sourceCode}
\lstinputlisting{../../C/Section1/Section1_1.c}

\end{document}