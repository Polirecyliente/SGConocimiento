
%   Basic syntax

%T% compile a tex file into PDF, in a shell terminal, with pdflatex
% pdflatex file1.tex

%T% a Latex file starts with the definition of the document class
%T% with \documentclass{docClass1}
\documentclass{article}

%T% Latex commands have the form \command[optionalArg]{mandatoryArg}
%T% set title, date, author with \title{Title1} \date{yyyy-mm-dd} 
%T% \author{Name1} in the preamble
\title{Section1\_1Title}
\date{2020-01-30}
\author{Juli Tovar}

%T% make @ at symbol be used to point to metadata with \makeatletter
\makeatletter
%T% use single metadata like title, date, and author with \@metadata
\let\newti\@title
%T% end this use of the @ at symbol with \makeatother
\makeatother

%T% the document itself is enclosed in \begin{document}, \end{document}
%T% this type of commands act as tags that enclose an environment
%T% document is the root tag
\begin{document}
  text1Line1 w2Line1
  stillLine1 endL1

%T% make a page with title, date, author with \maketitle
  \maketitle

  The title is \newti

%T% stop numbering pages with \pagenumbering{gobble}
  \pagenumbering{gobble}

%T% start a new page with \newpage
  \newpage
  Page 3

%T% restart numbering pages with \pagenumbering{arabic}
  \pagenumbering{arabic}
\end{document}